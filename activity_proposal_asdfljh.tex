\documentclass[a4paper, 11pt]{article}

\usepackage{kotex} % Comment this out if you are not using Hangul
\usepackage{fullpage}
\usepackage{hyperref}
\usepackage{amsthm}
\usepackage[numbers,sort&compress]{natbib}

\theoremstyle{definition}
\newtheorem{exercise}{Exercise}

\begin{document}
%%% Header starts
\noindent{\large\textbf{IS-521 Activity Proposal}\hfill
                \textbf{Jiho Lee}} \\
         {\phantom{} \hfill \textbf{asdfljh}} \\
         {\phantom{} \hfill Due Date: April 15, 2017} \\
%%% Header ends

\section{Activity Overview}

본 acitivity에서는 libpcap을 이용하여 NIDS(Network Intrusion Detection System) 구축하게 될 것이다. Pcap은 네트워크 트래픽을 캡쳐하기 위한 라이브러리이다. 다양한 네트워크 모니터링 소프트웨어 및 
IDS(e.g. tcpdump, wireshark, snort 등)가 pcap을 사용하고 있고 Unix 기반 시스템에서는 libpcap~\cite{libpcap} 라이브러리를 이용하여 pcap API를 사용할 수 있다. Pcap API는 C로 구현되어 있어 C 프로그래밍을 할 줄 안다면 누구나 자신만의 프로토콜 분석장비, 네트워크 모니터, IDS 등을 구현할 수 있다.

이 activity에서는 기본 네트워크 모니터 종류 중 하나인 NIDS를 구현해보는 것이 목표이다. NIDS를 구현하기 위해서는 모든 네트워크 트래픽을 확인해야 하므로 모니터링 기능이 필수적이며, 추가적으로 주어진 룰을 기반으로 특정 IP, PORT 혹은 payload에 대한 detection logging 기능이 구현되어야 한다. 이 acitivity를 통해, 패킷 캡쳐 기반의 다양한 소프트웨어의 동작 원리를 이해할 수 있게 되고, 네트워크 패킷 캡쳐/트래픽 관리 측면에서의 나만의 Swiss Army Knife를 가질 수 있게 된다!

\section{Exercises}

Overview에서는 IDS에 대한 설명만 해놓았지만, 사실 여러분들이 구현해야할 중요한 프로그램이 하나 더 있다. 이는, 본 activity의 테스트 환경과 함께 설명하겠다.

본 activity는 2개의 Client와 1개의 Server를 가정하여 이뤄질 것이다.
\\Client가 통신하게 될 Server에는 여러분이 만들게 될 IDS와 함께 vulnet이 올라갈 것이다. Vulnet은 이전 activity에서 제공되었던 vulnet을 사용할 것이지만 두 가지 조건을 만족할 것이다: (1) vulnet에서 사용되는 username과 password는 고정되어 있다. - 사실 큰 상관없지만 편의상 위의 가정을 기반으로 서술하겠습니다. (2) 가장 최근에 업데이트 된 vulnet~\cite{updatedvulnet}에서는 로그인된 shell로 입력된 문자열을 검사하는 부분이 있지만, 이번 vulnet에는 빠져있다. 여러분이 만들 IDS를 통해 vulnet을 이용한 공격이 발생했음을 탐지하면 된다.
\\Server는 2개의 Client 중 특정 Client로부터 오는 패킷을 아예 drop하게 될 것이다.(by iptables.) 이 Activity에서는 Ban이 된 Client에서 Ban되지 않은 Client ip로 spoofing하여 Server와 통신을 할 것이다. 당연히 Server는 spoofing된 ip로 응답을 보낼테고, Ban되지 않은 Client는 Ban이 된 Client로 그 응답을 포워딩해준다고 가정하면 ip spoofing을 하고도 통신할 수 있다. (* Exercise를 위해 이상한 가정을 넣었습니다.) 이를 위해 위에서 말했듯이 여러분은 IDS외에도 ip spoofing해서 Server에 보내는 프로그램 및 응답을 포워딩해주는 프로그램을 구현해야 한다.

정리하면, 이번 activity에서 만들어야 할 프로그램은 크게 두 가지이다: (1) IDS, (2) RAW socket을 이용한 TCP 통신 프로그램(IP spoofing 지원, 응답 포워딩 프로그램 포함).

\begin{exercise}

  IDS
  
  이번 activity에서 만들 IDS가 가져야하는 기능은 크게 3가지이다: (1) 특정 IP/PORT(IP로부터 들어오는 PORT로 들어오는) 패킷 탐지, (2) Payload 중 특정 문자열을 포함하는 패킷 탐지, (3) DoS(Denial of Service) 공격 탐지.
  \begin{enumerate}
    \item IDS에는 argument로 룰이 주어지는 데, 룰에는 탐지할 특정 IP/PORT 그리고 Payload 중 특정 문자열 등에 대한 정보가 주어질 것이다.
    \item 여러분이 만들 IDS는 주어지는 룰 파일을 파싱하여 조건을 만족하는 특정 패킷에 대한 정보를 파일 형태로 logging해야 한다. 룰에 걸리는 패킷의 경우에만 탐지된 조건과 함께 해당 패킷에 대한 정보(시간, src ip, 패킷 길이, 패킷 플래그 등)를 logging한다.
    \item 본 activity에서는 편의상 초당 10000개 이상의 패킷이 하나의 src ip에서 왔을 경우라고 정의하겠다. 이 경우, DoS가 발생한 시간과 함께 src ip를 logging한다.
  \end{enumerate}  

\end{exercise}

\begin{exercise}

  RAW socket을 이용한 TCP 통신 프로그램 - ip spoofing 사용.
  
  IP spoofing을 지원하는 RAW socket을 이용한 TCP 통신 프로그램을 구현해야 한다. RAW socket 프로그래밍에 대한 튜토리얼 자료~\cite{rawsocketexample}는 인터넷에 매우 많다! 이 프로그램은 Ban당한 Client에서 실행될 것이고, 직접 패킷을 보낼 ip, spoofing할 ip, 응답을 포워딩 받을 port를 argument형태로 입력받게 된다. 직접 패킷을 받게 될 Ban당하지 않은 Client에서는 포워딩 해줄 ip 및 port를 argument형태로 입력받는 프로그램이 실행될 것이다.

\end{exercise}

\section{Expected Solutions}

직접 나만의 IDS를 구현하기 위해서는 기존 IDS 소프트웨어에 대한 동작 원리를 학생들이 자세히 공부해야 할 것이다. 물론, IDS와 libpcap에 대한 코드 및 설명이 인터넷에 많이 있기 때문에 난이도가 매우 높은 과제는 아니라고 예상된다. IDS 구현에 있어서 특별한 제한을 두는 것이 아니기 때문에 다양한 방식으로 구현을 할 수 있지만 룰에 맞는 패킷을 모두 탐지할 수 있다면 큰 상관은 없을 듯 하다.
\\RAW socket 기반 TCP 통신 프로그램 또한, 인터넷이나 책 등에 많은 자료가 있어 공부하고 구현하는 작업에서 큰 어려움이 없을 것이라고 생각된다. 하지만, TCP 통신을 하는 코드가 sequence number와 acknowledgement number를 검사하고 알맞게 보내주는 루틴이 필요하다는 것을 인지해야하며 Ban되지 않은 Client로 Server가 패킷을 보낼 때 이에 대한 응답을 커널이 자동으로 보내지 않도록 처리해줘야 한다. (* iptables을 이용하면 쉽게 막을 수 있다.) 구현을 직접 하는 도중에 위와 같은 디테일한 문제를 직면하게 되고 이에 대한 많은 고민과 여러 자료를 찾아볼 것으로 예상된다.
\\두 프로그램 다 구현을 해보면, 네트워크 프로그래밍 실력을 키우는 데 있어서 크게 도움이 될 것이다. 또한, 확장 및 튜닝하기 좋은 프로그램(Swiss Army Knife)이기 때문에 다른 네트워크 문제를 해결할 때에도 사용할 수 있을 것이다.
\\(* 구현한 IDS는 Server에서 실행될 것이고, RAW socket 기반 TCP 통신 프로그램은 Client에서 실행되므로, 두 학생의 프로그램을 가지고 테스트해볼 수도 있을 것이다.)

\bibliography{references}
\bibliographystyle{plainnat}

\end{document}
